%%%%%%%%%%%%%%%%%%%%%%%%%%%%%%%%%%%%%%%%%%%%%%%%%%%%%%%%%%%%%%%%%%%%%%%%%%%%%%%%
%%%%%%%%%%%%%%%%%%%%%%%%%%%%%%%%%%%%%%%%%%%%%%%%%%%%%%%%%%%%%%%%%%%%%%%%%%%%%%%%
%%                                                                            %%
%% opintnaytepohja.tex versio 3.20 (2018/08/31)                               %%
%% Opinnäytepohja käytettäväksi aaltothesis.sty (versio 3.20) -tyylitiedoston %%
%% kanssa.                                                                    %%
%% Toimiakseen paketti tarvitsee pdfx.sty v. 1.5.84 (2017/05/18) tai uudempi. %%
%% The LaTeX template file to be used with the aaltothesis.sty (version 3.20) %%
%% style file.                                                                %%
%% This package requires pdfx.sty v. 1.5.84 (2017/05/18) or newer.            %%
%%                                                                            %%
%% This is licensed under the terms of the MIT license below.                 %%
%%                                                                            %%
%% Written by Luis R.J. Costa.                                                %%
%% Currently developed at the Learning Services of Aalto University School of %%
%% Electrical Engineering by Luis R.J. Costa since May 2017.                  %%
%%                                                                            %%
%% Copyright 2017-2018, by Luis R.J. Costa, luis.costa@aalto.fi,              %%
%% Copyright 2017-2018 Swedish translations in aaltothesis.cls by Elisabeth   %%
%% Nyberg, elisabeth.nyberg@aalto.fi and Henrik Wallén,                       %%
%% henrik.wallen@aalto.fi.                                                    %%
%% Copyright 2017-2018 Finnish documentation in the template opinnatepohja.tex%%
%% by Perttu Puska, perttu.puska@aalto.fi, and Luis R.J. Costa.               %%
%% Copyright 2018 English template thesistemplate.tex by Luis R.J. Costa.     %%
%% Copyright 2018 Swedish template kandidatarbetsbotten.tex by Henrik Wallen. %%
%%                                                                            %%
%% Permission is hereby granted, free of charge, to any person obtaining a    %%
%% copy of this software and associated documentation files (the "Software"), %%
%% to deal in the Software without restriction, including without limitation  %%
%% the rights to use, copy, modify, merge, publish, distribute, sublicense,   %%
%% and/or sell copies of the Software, and to permit persons to whom the      %%
%% Software is furnished to do so, subject to the following conditions:       %%
%% The above copyright notice and this permission notice shall be included in %%
%% all copies or substantial portions of the Software.                        %%
%% THE SOFTWARE IS PROVIDED "AS IS", WITHOUT WARRANTY OF ANY KIND, EXPRESS OR %%
%% IMPLIED, INCLUDING BUT NOT LIMITED TO THE WARRANTIES OF MERCHANTABILITY,   %%
%% FITNESS FOR A PARTICULAR PURPOSE AND NONINFRINGEMENT. IN NO EVENT SHALL    %%
%% THE AUTHORS OR COPYRIGHT HOLDERS BE LIABLE FOR ANY CLAIM, DAMAGES OR OTHER %%
%% LIABILITY, WHETHER IN AN ACTION OF CONTRACT, TORT OR OTHERWISE, ARISING    %%
%% FROM, OUT OF OR IN CONNECTION WITH THE SOFTWARE OR THE USE OR OTHER        %%
%% DEALINGS IN THE SOFTWARE.                                                  %%
%%                                                                            %%
%%                                                                            %%
%%%%%%%%%%%%%%%%%%%%%%%%%%%%%%%%%%%%%%%%%%%%%%%%%%%%%%%%%%%%%%%%%%%%%%%%%%%%%%%%

\documentclass[finnish, 12pt, a4paper, elec, utf8, a-1b, online]{aaltothesis}
%\documentclass[finnish, 12pt, a4paper, elec, utf8, a-1b]{aaltothesis}

\usepackage{graphicx}

\usepackage{amsfonts, amssymb, amsbsy, amsmath}

\usepackage{biblatex}
\addbibresource{refs.bib}

\degreeprogram{Automaatio- ja informaatioteknologia}

\major{Informaatioteknologia}

\code{ELEC3015}

\univdegree{BSc}

\thesisauthor{Aapo Kiiso}

\thesistitle{Verkkosivuston ulkoasun personointi}

\place{Espoo}

\date{xx.xx.2022}

\supervisor{Titteli Samuli Aalto}

\advisor{TkT Markku Laine}

\uselogo{aaltoBlue}{''}

\keywords{avainsana 1\spc{}avainsana 2\spc{}}

\thesisabstract{
}

\copyrighttext{Copyright \noexpand\copyright\ \number\year\ \ThesisAuthor}
{Copyright \copyright{} \number\year{} \ThesisAuthor}

\begin{document}

\makecoverpage{}

\makecopyrightpage{}

\begin{abstractpage}[finnish]
\end{abstractpage}

\thesistableofcontents{}

\mysection{Käsitteet ja lyhenteet}

\subsection*{Käsitteet}

\begin{tabular}{ll}
mukauttaminen & kohdentaminen käyttäjälle tai käyttäjäryhmälle sopivaksi \\
personointi & kerättyyn tietoon perustuva automaattinen mukauttaminen \\
kustomointi & valintoihin ja asetuksiin perustuva mukauttaminen
\end{tabular}

\subsection*{Lyhenteet}

\begin{tabular}{ll}
CSS & Cascading Style Sheets \\
HTML & HyperText Markup Language \\
RWD & Responsive Web Design
\end{tabular}

\cleardoublepage{}

\section{Johdanto}

Ihmisen ja tietokoneen välinen vuorovaikutus on ollut tietojenkäsittelytieteessä
tutkimuksen kohteena henkilökohtaisen tietokoneen läpimurrosta 1970- ja
1980-lukujen vaihteesta lähtien~\cite{10.1145/800178.810088}. Yksi tutkimuksen
tavoitteista on ollut löytää menetelmiä mukauttaa tietokoneen ohjelmistoja
kullekin käyttäjälle sopivaksi.

Ohjelmistojen mukauttaminen voidaan jakaa karkeasti kahteen ryhmään:
Kustomoinnilla tarkoitetaan käyttäjän itse tekemää mukauttamista, personoinnilla
automaattista mukauttamista käyttäjästä kerätyn tiedon
perusteella.~\cite{10.1108/03090560710737534}

Aikaisessa tutkimuksessa tutkittiin nimenomaan ohjelmistojen mukauttamista
kustomoinnin kautta, eli muun muassa tarjoamalla käyttäjälle asetuksia
tarpeettomien toimintojen piilottamiseen. Ajan kuluessa tutkimuksen painopiste
on siirtynyt kustomoinnista personoinnin puolelle~\cite{viite?}.

Alkuaikoina ohjelmistojen jakelu fyysisten levykkeiden ja myöhemmin CD-levyjen
muodossa hankaloitti personointiin liittyvän teknologian tutkimusta ja
kehitystä, sillä ohjelmistopäivitysten jakelu käyttäjille oli hidasta ja
kallista. 1990-luvulla yleistyneitä internet- eli verkkosivustoja ei tarvitse
perinteisten ohjelmistojen tapaan jaella käyttäjille fyysisessä muodossa, vaan
internetin avulla voidaan tarjoilla palvelimelta aina uusin versio
verkkosivustosta. Internetin mahdollistama palveluntarjoajan ja loppukäyttäjän
välinen reaaliaikainen vuorovaikutus johti myös osaltaan personointia
hyödyntävän liiketoiminnan, kuten verkkokaupankäynnin ja sosiaalisen median
kehittymiseen. Internetin käytön leviäminen arkielämään 1990- ja 2000-lukujen
vaihteessa kiihdytti personointiin kohdistuvaa tutkimusta ja
kehitystä edelleen~\cite{10.1108/03090560710737534}.

Vaikka verkkosivustojen personointia on tutkittu verrattain pitkään, käytännön
hyödyntäminen toimialalla on vielä harvinaista~\cite{viite?}.
Verkkosivuston ulkoasu suunnitellaan nykyäänkin lähtökohtaisesti ihmisen toimesta,
ja myös ulkoasusuunnitelman tulkitseminen ja ohjelmoiminen lopulliseksi
verkkosivustoksi on manuaalista työtä. Käyttäjille tai edes käyttäjäryhmille ei
siis ole kustannustehokasta suunnitella personoituja versioita sivustoista.
Yhden ja saman version jakelu kaikille verkkosivuston käyttäjille ei ole
kuitenkaan aina optimaalista, sillä käyttäjillä on usein eri tarpeita muun
muassa kulttuuritaustasta ja iästä riippuen~\cite{viite?}.

Työn tarkoitus on selvittää mitä menetelmiä verkkosivuston ulkoasun eri osien
personointiin on olemassa. Tarkasteltavia osia ovat muun muassa verkkosivuston
asettelu, kirjasin- ja värivalinnat sekä valikot. Työ tarkastelee myös eri
personointimenetelmien toimintatapoja. Monet tarkasteltavista menetelmistä
perustuvat matemaattisen optimointiin, mutta hyödyntävät myös
käyttöliittymäsuunnittelun heuristiikkoja kuten tekstin luettavuuskriiterejä
tuloksissaan. Oma lukunsa ovat koneoppimiseen perustuvat menetelmät, joiden
opetusdatana on käytetty esimerkiksi olemassa olevia verkkosivustoja. Työ myös
vertailee menetelmien käyttöönoton kustannuksia saavutettaviin hyötyihin, ja
pyrkii tätä kautta löytämään matalan kynnyksen persononointimenetelmät, joista
saa suurimman hyödyn.

\clearpage

\section{Aikaisempi tutkimus}

Tarvitaanko tätä? Tulee mun mielestä ilmi seuraavissa kappaleissa.
\\
- Personalisoinnin tutkimuksen historia tähän päivään?
\\
- Personalisoinnin huonot puolet? Pohdintaa tietosuojasta ja siitä että
seurataan ihmisiä ja kerätään niistä tietoa
\\
- Web-suunnittelu esittely yleisesti? suunnittelijan työkalut
\\
- Web-kehitys esittely yleisesti? HTML/CSS/JS, perinteiset floatti-leiskat, flexbox, css grid
\\

\clearpage

\section{Tarve personoinnille}

\clearpage

\section{Asettelun personointi}

Sivun asettelu määrittää verkkosivun perusrakenteen ja sivun sisällön
keskinäisen järjestyksen. Tyypillisesti verkkosivuston yksittäisellä sivulla on
vähintään ylätunniste (engl.\ \textit{header}), sisältöalue (engl.\
\textit{content area}) ja alatunniste (engl.\ \textit{footer}). Ylätunniste ja
alatunniste pysyvät yleensä suurin piirtein muuttumattomina sivustolla
navigoidessa, mutta sisältöalueen sisältö luonnollisesti vaihtuu sivun mukaan.
Ylätunniste ja alatunniste ovat valikonomaisia sivuston osia, joita käsitellään
luvussa~\ref{menus}. Ylätunniste sijaitsee aina sivun alussa ja alatunniste
sivun lopussa. Niiden asettelun personointi ei ole mielekästä, joten niitä ei
tarkastella enempää tässä luvussa.

Sisältöalueen asettelu perustuu nykyään lähtökohtaisesti ruudukkorakenteeseen
(engl.\ \textit{grid layout}). Ruudukko koostuu sivustosta ja päätelaitteen
näyttökoosta riippuen yleensä n. \ 1--16 sarakkeesta (engl.\ \textit{column}),
jotka on erotettu toisistaan vakiosuuruisella marginaalilla (engl.\
\textit{margin}). Sarakkeiden leveys, joka on kääntäen verrannollinen
sarakkeiden määrään, perustuu suhteelliseen osaan saatavilla olevasta
ruudukkoalueen (engl.\ \textit{grid container}) leveydestä, ja sarakkeiden
leveys muuttuu yhdessä ruudukkoalueen kanssa. Ruudukkoalueelle sijoittuvat
elementit (engl.\ \textit{element}), kuten tekstipalstat ja kuvat, voivat kattaa
yhden tai useamman sarakkeen leveyden. Elementtien leveys määritetään kuitenkin
nimenomaan sarakkeiden kautta, eivätkä ne voi vapaasti laajentua leveyssuunnassa
niiden sisällön kasvaessa. Sen sijaan elementit laajenevat tarvittaessa
pituussuunnassa. Tyypillisesti verkkosivut eivät mahdukaan kokonaan kerralla
näytölle, vaan niitä pitää vierittää alaspäin jolloin lisää sisältöä tulee
näkyville.

\subsection{Responsiivinen web-suunnittelu}

Ennen 2010-lukua asettelu tehtiin lähtökohtaisesti staattisesti, tarkoittaen
että sivun sisältöalueella oli ennalta määrätty leveys. Leveys oli yleensä
maksimissaan 1024 pikseliä, joka oli senaikaiset päätelaitteet huomioon ottaen sopiva
koko~\cite{viite?}. Sisältöalueen leveys ei muuttunut päätelaitteen mukaan. Sivun suunnittelu
ja toteutus oli verrattain helppoa, koska hankalista asettelun rajatapauksista
ei tarvinnut huolehtia. Toisaalta staattisesti rakennetut sivustot alkoivat olla
hankalia käyttää älypuhelinten yleistyessä, sillä sivustojen käyttö pienellä
näyttökoolla vaati zoomausta sisällön erottamiseksi.

Nykypäivänä on tyypillistä, että sivuston asettelu mukautuu päätelaitteen
näyttökokoon. Näyttökokoon perustuvaa asettelun mukautumista kutsutaan
responsiiviseksi web-suunnitteluksi (engl.\ \textit{responsive web design}),
joskin termi kattaa myös muita mukauttamisperusteita, kuten paperille
tulostuksen. Sivuston perustuessa ruudukkorakenteeseen asettelu mukautuu
automaattisesti näyttökoon muutoksiin, sillä sarakkeiden leveys on määritetty
suhteessa ruudukkoalueeseen. Asettelua voi kuitenkin parantaa entisestään
lisäämällä sarakkeiden määrää näyttökoon kasvaessa, ja vähentämällä sarakkeiden
määrää näyttökoon pienentyessä. Tällä tavalla esimerkiksi työpöytälaiteella
vierekkäin näkyvät elementit voidaan mobiililaitteella asetella allekkain,
jolloin ne ovat pienelläkin näyttökoolla yhtä suuria kuin työpöytälaiteella.

Asettelun muuttaminen näyttökoon muuttuessa tapahtuu leveyteen perustuvien
pysäytyspisteiden (engl.\ \textit{breakpoint}) avulla. Pysäytyspisteet
määritetään \textit{media queries} -tekniikan~\cite{Rivoal:12:MQ} avulla, joka
mahdollistaa muun muassa CSS-tyylien kohdentamisen päätelaitteen ominaisuuksien
mukaan. Pysäytyspisteisiin pohjautuva asettelun mukauttaminen on yleinen ja
helppo tapa mukauttaa sivuston asettelua eri päätelaitteille. Pysäytyspisteiden
avulla asettelu mukautetaan yleensä vähintään niin, että puhelimille,
tableteille ja työpöytälaitteille on kullekin oma asettelunsa.

\subsection{Personointi suunnitteluvaiheessa}

Personointimenetelmät voivat toimia suunnittelijan työkalujen kontekstissa.
Tällöin ne on helpompi ottaa käyttöön, koska työkalut on lähtökohtaisesti samat
kaikilla ja niihin integroituminen voidaan lisäosa-tyyppisten ratkaisujen
kautta. Nykyään suosittuja verkkosivustojen suunnitteluun käytettäviä
ohjelmistoja ovat muun muassa Sketch, Figma ja Adobe XD, joihin kaikkiin on
mahdollista rakentaa lisäosia. Perinteisempiä työkaluja edustavat Adoben
tuoteperheen Photoshop ja Illustrator, joiden käyttö web-suunnittelussa on
vähentynyt huomattavasti.

Yksi suunnittelijoille suunnattu apukeino on Pang et al.~\cite{Pang2016}
esittelemä menetelmä verkkosivuston asettelun optimoimiseen käyttäjien
silmänliikkeistä kerätyn datan perusteella. Suunnittelija kertoo työkalulle
tavoittelemansa elementtien lukujärjestyksen, ja työkalu tuottaa suunnitelman
verkkosivusta jossa elementtien asettelu on optimoitu annetun lukujärjestyksen
mukaan. Pangin työkalu ei integroidu suoraan suunnittelijan työkaluihin, mutta
vastaavan kaupallisen ratkaisun tuominen yleisimpiin työkaluihin olisi varmasti
tervetullut apu suunnittelijoille.

Suunnitteluvaiheen optimointimenetelmien kautta on hankala personoida
verkkosivuja, koska toteutusvaihe vaatii edelleen manuaalista työtä. Personointi
jääkin responsiivisen web-suunnittelun tapaan melko yleiselle, isoille
käyttäjäryhmille suunnatulle tasolle.

\subsection{Asettelun laskennallinen personointi}

Responsiivisen web-suunnittelun haasteena on se, että se täytyy ottaa huomioon
jo sivuston suunnittelu- ja toteutusvaiheessa. Suunnittelijan tulee prosessin
aikaisessa vaiheessa päättää, mitkä pysäytyspisteet sivustolla on käytössä.
Pysäytyspisteiden muuttaminen myöhemmin suunnittelun tai toteutuksen aikana on
hyvin työlästä, sillä pysäytyspisteet määrittävät asettelun perustan ja niiden
muuttaminen vaikuttaa siten kaikkeen muuhun.

Responsiivinen web-suunnittelu ei ole personointia sanan varsinaisessa
merkityksessä, sillä sivuston responsiviisuus on määritelty jo
suunnitteluvaiheessa eikä aidosti mukaudu yksittäisen käyttäjän tarpeisiin.
Responsiivisen web-suunnittelun kautta tapahtuva asettelun personointi perustuu
vain käyttäjän laitteen karkeaan kategorisointiin, yleensä joko mobiili-,
tabletti- tai työpöytälaitteeksi.

Näistä lähtökohdista tutkimusta on kohdistunut laskennallisten menetelmien
hyödyntämiseen responsiivisessa web-suunnittelussa. Verkkosivusto on
pohjimmiltaan rakenteellista sisältöä, joten sitä voidaan käsitellä
koneellisesti. Jos responsiivinen web-suunnittelu kyetään automatisoimaan,
vähentää se suunnittelijan ja kehittäjän työtaakkaa huomattavasti. Tällöin
riittää että sivustosta suunnitellaan vain mobiiliversio (ns. \textit{mobile
first} web-suunnittelu), josta voidaan laskennallisten menetelmien avulla
laajentaa muille laitteille sopivat versiot.

Aalto-yliopiston \href{https://userinterfaces.aalto.fi/}{User Interfaces
-tutkimusryhmä} kehitti vuonna 2020 \textit{Layout as a
Service}-alustakonseptin~\cite{laine2020_laas} ja myöhemmin vuonna 2021 siihen
pohjautuvan \textit{Computational responsive web design
(C-RWD)}-teknologiakonseptin~\cite{laine2021responsive}. Tutkimusryhmän C-RWD
-teknologia kykenee automatisoimaan sivuston mukauttamisen eri päätelaitteille
ja näyttöko'oille. Teknologia seuraa sivuston vieralijoiden käyttäytymiseen
liittyviä tietoja, kuten vierailuaikaa kullakin sivulla ja mitä sisältöjä on
klikattu. C-RWD parsii sivun HTML-rakennetta ja kykenee kerättyjen tietojen
avulla optimoimaan siitä käyttäjälle personoidun version. Personoitu versio
sivustosta tarjoillaan käyttäjälle vanhan sivuston sijasta. Käyttäjälle
tärkeiksi arvioituja sisältöjä voidaan esimerkiksi nostaa ylemmäs sivulla,
jolloin ne näkyvät käyttäjälle heti sivun latautuessa.

C-RWD laajentaa asettelun laskennallista suunnittelua nimenomaan responsiivisen
web-suunnittelun osalta. Aiempaa tutkimusta edustaa mm.
SUPPLE~\cite{10.1145/964442.964461}, joka tutki sivun asettelua
optimointiongelmana. SUPPLE kuitenkin vaati suunnittelijaa määrittämään
asettelun rajoitteet, kuten päätelaitteen näyttökoon ja käyttötavan. C-RWD
päättelee optimointiin tarvittavat rajoitteet itse seuraamalla käyttäjän
käyttäytymistä.

\section{Ilmeen personointi}

Fontit, värit, jne.

\section{Valikoiden personointi}\label{menus}

Valikkojen uudelleenjärjestely jne.

\section{Muu personointi}

Lyhyesti vaan. Kannattaakohan ees pitää?

\subsection{Sisältö}

Content-based filtering, etc.

\subsection{Lomakkeet}

Lomakkeiden personointi.

\subsection{Saavutettavuus}

Apuelementit (tooltips), aria, jne.

\clearpage

\section{Personointimenetelmien toimintatavat}

- Matemaattinen optimointi: integer programming
\\
- Käyttöliittymäsuunnittelun heuristiikat: fonttikoko hyvä 16px, väri-säännöt
\\
- Koneoppiminen: tähän kuva ai->ml->deep ml
\\
- Suunnittelijan input personointiin

\clearpage

\section{Personointimenetelmien vertailu}

Kuinka personoinnin menetelmät voidaan luokitella niin, että ne ovat helposti
vertailtavissa verkkosivuston kehittäjän näkökulmasta?

\subsection{Käyttöönoton monimutkaisuus ja hinta}

Kuinka kallista / vaikeaa ne on ottaa käyttöön?

\subsection{Vaikuttavuus ja hyöty}

Kuinka suuri positiivinen vaikutus niillä on?

\subsection{Nykyinen käyttö toimialalla}

Kuinka laajasti ne ovat käytössä toimialalla?

\subsection{Tulevaisuuden näkymät}

Onko jatkokehitystä, "trendaako"?

\clearpage

\section{Yhteenveto}

Summa summarum parhaat menetelmät, jotka voi ottaa käyttöön matalalla
kynnyksellä.

\clearpage

\thesisbibliography{}
\printbibliography{}

\clearpage
\thesisappendix{}

\section{Esimerkki liitteestä\label{LiiteA}}

- Kuva tyypillisestä verkkosivusta jossa korostettu työn kannalta relevantit
osat, kuten rakenne, ilme, sisältö, saavutettavuus.

\end{document}
