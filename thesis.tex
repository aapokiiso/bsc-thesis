%%%%%%%%%%%%%%%%%%%%%%%%%%%%%%%%%%%%%%%%%%%%%%%%%%%%%%%%%%%%%%%%%%%%%%%%%%%%%%%%
%%%%%%%%%%%%%%%%%%%%%%%%%%%%%%%%%%%%%%%%%%%%%%%%%%%%%%%%%%%%%%%%%%%%%%%%%%%%%%%%
%%                                                                            %%
%% opintnaytepohja.tex versio 3.20 (2018/08/31)                               %%
%% Opinnäytepohja käytettäväksi aaltothesis.sty (versio 3.20) -tyylitiedoston %%
%% kanssa.                                                                    %%
%% Toimiakseen paketti tarvitsee pdfx.sty v. 1.5.84 (2017/05/18) tai uudempi. %%
%% The LaTeX template file to be used with the aaltothesis.sty (version 3.20) %%
%% style file.                                                                %%
%% This package requires pdfx.sty v. 1.5.84 (2017/05/18) or newer.            %%
%%                                                                            %%
%% This is licensed under the terms of the MIT license below.                 %%
%%                                                                            %%
%% Written by Luis R.J. Costa.                                                %%
%% Currently developed at the Learning Services of Aalto University School of %%
%% Electrical Engineering by Luis R.J. Costa since May 2017.                  %%
%%                                                                            %%
%% Copyright 2017-2018, by Luis R.J. Costa, luis.costa@aalto.fi,              %%
%% Copyright 2017-2018 Swedish translations in aaltothesis.cls by Elisabeth   %%
%% Nyberg, elisabeth.nyberg@aalto.fi and Henrik Wallén,                       %%
%% henrik.wallen@aalto.fi.                                                    %%
%% Copyright 2017-2018 Finnish documentation in the template opinnatepohja.tex%%
%% by Perttu Puska, perttu.puska@aalto.fi, and Luis R.J. Costa.               %%
%% Copyright 2018 English template thesistemplate.tex by Luis R.J. Costa.     %%
%% Copyright 2018 Swedish template kandidatarbetsbotten.tex by Henrik Wallen. %%
%%                                                                            %%
%% Permission is hereby granted, free of charge, to any person obtaining a    %%
%% copy of this software and associated documentation files (the "Software"), %%
%% to deal in the Software without restriction, including without limitation  %%
%% the rights to use, copy, modify, merge, publish, distribute, sublicense,   %%
%% and/or sell copies of the Software, and to permit persons to whom the      %%
%% Software is furnished to do so, subject to the following conditions:       %%
%% The above copyright notice and this permission notice shall be included in %%
%% all copies or substantial portions of the Software.                        %%
%% THE SOFTWARE IS PROVIDED "AS IS", WITHOUT WARRANTY OF ANY KIND, EXPRESS OR %%
%% IMPLIED, INCLUDING BUT NOT LIMITED TO THE WARRANTIES OF MERCHANTABILITY,   %%
%% FITNESS FOR A PARTICULAR PURPOSE AND NONINFRINGEMENT. IN NO EVENT SHALL    %%
%% THE AUTHORS OR COPYRIGHT HOLDERS BE LIABLE FOR ANY CLAIM, DAMAGES OR OTHER %%
%% LIABILITY, WHETHER IN AN ACTION OF CONTRACT, TORT OR OTHERWISE, ARISING    %%
%% FROM, OUT OF OR IN CONNECTION WITH THE SOFTWARE OR THE USE OR OTHER        %%
%% DEALINGS IN THE SOFTWARE.                                                  %%
%%                                                                            %%
%%                                                                            %%
%%%%%%%%%%%%%%%%%%%%%%%%%%%%%%%%%%%%%%%%%%%%%%%%%%%%%%%%%%%%%%%%%%%%%%%%%%%%%%%%

\documentclass[finnish, 12pt, a4paper, elec, utf8, a-1b, online]{aaltothesis}
%\documentclass[finnish, 12pt, a4paper, elec, utf8, a-1b]{aaltothesis}

\usepackage{graphicx}

\usepackage{amsfonts, amssymb, amsbsy, amsmath}

\usepackage{biblatex}
\addbibresource{refs.bib}

\degreeprogram{Automaatio- ja informaatioteknologia}

\major{Informaatioteknologia}

\code{ELEC3015}

\univdegree{BSc}

\thesisauthor{Aapo Kiiso}

\thesistitle{Verkkosivuston ulkoasun personointi}

\place{Espoo}

\date{xx.xx.2022}

\supervisor{Titteli Samuli Aalto}

\advisor{TkT Markku Laine}

\uselogo{aaltoBlue}{''}

\keywords{avainsana 1\spc{}avainsana 2\spc{}}

\thesisabstract{
}

\copyrighttext{Copyright \noexpand\copyright\ \number\year\ \ThesisAuthor}
{Copyright \copyright{} \number\year{} \ThesisAuthor}

\begin{document}

\makecoverpage{}

\makecopyrightpage{}

\begin{abstractpage}[finnish]
\end{abstractpage}

\thesistableofcontents{}

\mysection{Käsitteet ja lyhenteet}

\subsection*{Käsitteet}

\begin{tabular}{ll}
personointi & palvelun yksilöllistäminen yksilöstä kerätyn tiedon perusteella
\end{tabular}

\subsection*{Lyhenteet}

\begin{tabular}{ll}
CSS & Cascading Style Sheets \\
HTML & HyperText Markup Language
\end{tabular}

\cleardoublepage{}

\section{Johdanto}

Ihmisen ja tietokoneen välinen vuorovaikutus on ollut tietojenkäsittelytieteessä
tutkimuksen kohteena jo henkilökohtaisen tietokoneen läpimurrosta 1970- ja
1980-lukujen vaihteesta lähtien~\cite{10.1145/800178.810088}. Yksi tutkimuksen
tavoitteista on ollut löytää menetelmiä mukauttaa tietokoneen ohjelmistoja
kullekin käyttäjälle sopivaksi.

Ohjelmistojen mukauttaminen voidaan jakaa karkeasti kahteen ryhmään:
Kustomoinnilla tarkoitetaan käyttäjän itse tekemää mukauttamista, personoinnilla
sen sijaan automaattista mukauttamista käyttäjästä kerätyn tiedon
perusteella.~\cite{10.1108/03090560710737534}

Aikaisessa tutkimuksessa tutkittiin nimenomaan ohjelmistojen mukauttamista
kustomoinnin kautta, eli muun muassa tarjoamalla käyttäjälle asetuksia
tarpeettomien toimintojen piilottamiseen. Ajan kuluessa tutkimuksen painopiste
on siirtynyt kustomoinnista personoinnin puolelle.

Alkuaikoina ohjelmistojen jakelu fyysisten levykkeiden ja myöhemmin CD-levyjen
muodossa hankaloitti personointiin liittyvän teknologian tutkimusta ja
kehitystä, sillä ohjelmistopäivitysten jakelu käyttäjille oli hidasta ja
kallista. 1990-luvulla yleistyneitä internet- eli verkkosivustoja ei tarvite
perinteisten ohjelmistojen tapaan jaella käyttäjille fyysisessä muodossa, vaan
heille voidaan internetin avulla välittää palvelimelta aina uusin versio
verkkosivustosta. Internetin mahdollistama palveluntarjoajan ja loppukäyttäjän
välinen reaaliaikainen vuorovaikutus johti myös osaltaan personointia
hyödyntävän liiketoiminnan, kuten verkkokaupankäynnin ja sosiaalisen median
kehittymiseen. Internetin käytön leviäminen arkielämään 1990- ja 2000-lukujen
vaihteessa kiihdytti personointiin kohdistuvaa tutkimusta ja
kehitystä~\cite{10.1108/03090560710737534}.

Vaikka verkkosivustojen personointia on tutkittu verrattain pitkään, käytännön
hyödyntäminen toimialalla on edelleen harvinaista~\cite{viite puuttuu}.
Verkkosivuston ulkoasu suunnitellaan edelleen lähtökohtaisesti ihmisen toimesta,
ja myös ulkoasusuunnitelman tulkitseminen ja ohjelmoiminen lopulliseksi
verkkosivustoksi on manuaalista työtä. Käyttäjille tai edes käyttäjäryhmille ei
siis ole kustannustehokasta suunnitella personoituja versioita sivustoista.
Yhden ja saman version jakelu kaikille verkkosivuston käyttäjille ei ole
kuitenkaan aina optimaalista, sillä käyttäjillä on usein eri tarpeita muun
muassa kulttuuritaustasta ja iästä riippuen~\cite{viite puuttuu}.

Työn tarkoitus on selvittää mitä menetelmiä verkkosivuston eri osien
personointiin on olemassa. Tarkasteltavia osia ovat muun muassa verkkosivuston
rakenne, kirjasin- ja värivalinnat, valikot, sisältö ja saavutettavuus. Työ
tarkastelee myös eri personointimenetelmien toimintatapoja. Monet
tarkasteltavista menetelmistä perustuvat matemaattisen optimointiin, mutta
hyödyntävät myös käyttöliittymäsuunnittelun heuristiikkoja kuten tekstin
luettavuuskriiterejä tuloksissaan. Oma lukunsa ovat koneoppimiseen perustuvat
menetelmät, joiden opetusdatana on käytetty esimerkiksi olemassa olevia
verkkosivustoja. Työ myös vertailee menetelmien käyttöönoton kustannuksia
saavutettaviin hyötyihin, ja pyrkii tätä kautta löytämään matalan kynnyksen
persononointimenetelmät, joista saa suurimman hyödyn.

\clearpage

\section{Aikaisempi tutkimus}

Tarvitaanko tätä? Tulee mun mielestä ilmi seuraavissa kappaleissa.

\clearpage

\section{Verkkosivuston osien personointi}

Tässä käydään läpi kaikki eri verkkosivuston relevantit osa-alueet ja kuinka
niitä voidaan personoida.

\subsection{Sivun asettelu}

Sivuston rakenteen personointi.

\subsection{Ulkoasu ja ilme}

Fontit, värit, jne.

\subsection{Sisältö}

Sisällön korostaminen jne.

\subsection{Valikot}

Valikkojen uudelleenjärjestely jne.

\subsection{Lomakkeet}

Lomakkeiden personointi.

\subsection{Saavutettavuus}

Apuelementit (tooltips), aria, jne.

\clearpage

\section{Personointimenetelmien toimintatavat}

- Matemaattinen optimointi (integer programming?)
\\
- Käyttöliittymäsuunnittelun heuristiikat
\\
- Koneoppiminen
\\
- Suunnittelijan syöte / työkalut

\clearpage

\section{Personointimenetelmien luokittelu}

Kuinka personoinnin menetelmät voidaan luokitella niin, että ne ovat helposti
vertailtavissa verkkosivuston kehittäjän näkökulmasta?

\subsection{Käyttöönoton monimutkaisuus ja hinta}

Kuinka kallista / vaikeaa ne on ottaa käyttöön?

\subsection{Vaikuttavuus ja hyöty}

Kuinka suuri positiivinen vaikutus niillä on?

\subsection{Nykyinen käyttö toimialalla}

Kuinka laajasti ne ovat käytössä toimialalla?

\subsection{Tulevaisuuden näkymät}

Onko jatkokehitystä, "trendaako"?

\clearpage

\section{Yhteenveto}

Summa summarum parhaat menetelmät, jotka voi ottaa käyttöön matalalla
kynnyksellä.

\clearpage

\thesisbibliography{}
\printbibliography{}

\end{document}
