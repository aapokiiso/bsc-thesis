%%%%%%%%%%%%%%%%%%%%%%%%%%%%%%%%%%%%%%%%%%%%%%%%%%%%%%%%%%%%%%%%%%%%%%%%%%%%%%%%
%%%%%%%%%%%%%%%%%%%%%%%%%%%%%%%%%%%%%%%%%%%%%%%%%%%%%%%%%%%%%%%%%%%%%%%%%%%%%%%%
%%                                                                            %%
%% opintnaytepohja.tex versio 3.20 (2018/08/31)                               %%
%% Opinnäytepohja käytettäväksi aaltothesis.sty (versio 3.20) -tyylitiedoston %%
%% kanssa.                                                                    %%
%% Toimiakseen paketti tarvitsee pdfx.sty v. 1.5.84 (2017/05/18) tai uudempi. %%
%% The LaTeX template file to be used with the aaltothesis.sty (version 3.20) %%
%% style file.                                                                %%
%% This package requires pdfx.sty v. 1.5.84 (2017/05/18) or newer.            %%
%%                                                                            %%
%% This is licensed under the terms of the MIT license below.                 %%
%%                                                                            %%
%% Written by Luis R.J. Costa.                                                %%
%% Currently developed at the Learning Services of Aalto University School of %%
%% Electrical Engineering by Luis R.J. Costa since May 2017.                  %%
%%                                                                            %%
%% Copyright 2017-2018, by Luis R.J. Costa, luis.costa@aalto.fi,              %%
%% Copyright 2017-2018 Swedish translations in aaltothesis.cls by Elisabeth   %%
%% Nyberg, elisabeth.nyberg@aalto.fi and Henrik Wallén,                       %%
%% henrik.wallen@aalto.fi.                                                    %%
%% Copyright 2017-2018 Finnish documentation in the template opinnatepohja.tex%%
%% by Perttu Puska, perttu.puska@aalto.fi, and Luis R.J. Costa.               %%
%% Copyright 2018 English template thesistemplate.tex by Luis R.J. Costa.     %%
%% Copyright 2018 Swedish template kandidatarbetsbotten.tex by Henrik Wallen. %%
%%                                                                            %%
%% Permission is hereby granted, free of charge, to any person obtaining a    %%
%% copy of this software and associated documentation files (the "Software"), %%
%% to deal in the Software without restriction, including without limitation  %%
%% the rights to use, copy, modify, merge, publish, distribute, sublicense,   %%
%% and/or sell copies of the Software, and to permit persons to whom the      %%
%% Software is furnished to do so, subject to the following conditions:       %%
%% The above copyright notice and this permission notice shall be included in %%
%% all copies or substantial portions of the Software.                        %%
%% THE SOFTWARE IS PROVIDED "AS IS", WITHOUT WARRANTY OF ANY KIND, EXPRESS OR %%
%% IMPLIED, INCLUDING BUT NOT LIMITED TO THE WARRANTIES OF MERCHANTABILITY,   %%
%% FITNESS FOR A PARTICULAR PURPOSE AND NONINFRINGEMENT. IN NO EVENT SHALL    %%
%% THE AUTHORS OR COPYRIGHT HOLDERS BE LIABLE FOR ANY CLAIM, DAMAGES OR OTHER %%
%% LIABILITY, WHETHER IN AN ACTION OF CONTRACT, TORT OR OTHERWISE, ARISING    %%
%% FROM, OUT OF OR IN CONNECTION WITH THE SOFTWARE OR THE USE OR OTHER        %%
%% DEALINGS IN THE SOFTWARE.                                                  %%
%%                                                                            %%
%%                                                                            %%
%%%%%%%%%%%%%%%%%%%%%%%%%%%%%%%%%%%%%%%%%%%%%%%%%%%%%%%%%%%%%%%%%%%%%%%%%%%%%%%%

\documentclass[finnish, 12pt, a4paper, elec, utf8, a-1b, online]{aaltothesis}
%\documentclass[finnish, 12pt, a4paper, elec, utf8, a-1b]{aaltothesis}

\usepackage{graphicx}

\usepackage{amsfonts, amssymb, amsbsy, amsmath}

\usepackage{biblatex}
\addbibresource{refs.bib}

\degreeprogram{Automaatio- ja informaatioteknologia}

\major{Informaatioteknologia}

\code{ELEC3015}

\univdegree{BSc}

\thesisauthor{Aapo Kiiso}

\thesistitle{Verkkosivuston ulkoasun personointi}

\place{Espoo}

\date{xx.xx.2022}

\supervisor{Titteli Samuli Aalto}

\advisor{TkT Markku Laine}

\uselogo{aaltoBlue}{''}

\keywords{avainsana 1\spc{}avainsana 2\spc{}}

\thesisabstract{
}

\copyrighttext{Copyright \noexpand\copyright\ \number\year\ \ThesisAuthor}
{Copyright \copyright{} \number\year{} \ThesisAuthor}

\begin{document}

\makecoverpage{}

\makecopyrightpage{}

\begin{abstractpage}[finnish]
\end{abstractpage}

\thesistableofcontents{}

\mysection{Käsitteet ja lyhenteet}

\subsection*{Käsitteet}

\begin{tabular}{ll}
mukauttaminen & kohdentaminen käyttäjälle tai käyttäjäryhmälle sopivaksi \\
personointi & kerättyyn tietoon perustuva automaattinen mukauttaminen \\
kustomointi & valintoihin ja asetuksiin perustuva mukauttaminen
\end{tabular}

\subsection*{Lyhenteet}

\begin{tabular}{ll}
CSS & Cascading Style Sheets \\
HTML & HyperText Markup Language \\
RWD & Responsive Web Design
\end{tabular}

\cleardoublepage{}

\section{Johdanto}

Ihmisen ja tietokoneen välinen vuorovaikutus on ollut tietojenkäsittelytieteessä
tutkimuksen kohteena henkilökohtaisen tietokoneen läpimurrosta 1970- ja
1980-lukujen vaihteesta lähtien~\cite{10.1145/800178.810088}. Yksi tutkimuksen
tavoitteista on ollut löytää menetelmiä mukauttaa tietokoneen ohjelmistoja
kullekin käyttäjälle sopivaksi.

Ohjelmistojen mukauttaminen voidaan jakaa karkeasti kahteen ryhmään:
Kustomoinnilla tarkoitetaan käyttäjän itse tekemää mukauttamista, personoinnilla
automaattista mukauttamista käyttäjästä kerätyn tiedon
perusteella.~\cite{10.1108/03090560710737534}

Aikaisessa tutkimuksessa tutkittiin nimenomaan ohjelmistojen mukauttamista
kustomoinnin kautta, eli muun muassa tarjoamalla käyttäjälle asetuksia
tarpeettomien toimintojen piilottamiseen. Ajan kuluessa tutkimuksen painopiste
on siirtynyt kustomoinnista personoinnin puolelle~\cite{viite?}.

Alkuaikoina ohjelmistojen jakelu fyysisten levykkeiden ja myöhemmin CD-levyjen
muodossa hankaloitti personointiin liittyvän teknologian tutkimusta ja
kehitystä, sillä ohjelmistopäivitysten jakelu käyttäjille oli hidasta ja
kallista. 1990-luvulla yleistyneitä internet- eli verkkosivustoja ei tarvitse
perinteisten ohjelmistojen tapaan jaella käyttäjille fyysisessä muodossa, vaan
internetin avulla voidaan tarjoilla palvelimelta aina uusin versio
verkkosivustosta. Internetin mahdollistama palveluntarjoajan ja loppukäyttäjän
välinen reaaliaikainen vuorovaikutus johti myös osaltaan personointia
hyödyntävän liiketoiminnan, kuten verkkokaupankäynnin ja sosiaalisen median
kehittymiseen. Internetin käytön leviäminen arkielämään 1990- ja 2000-lukujen
vaihteessa kiihdytti personointiin kohdistuvaa tutkimusta ja
kehitystä edelleen~\cite{10.1108/03090560710737534}.

Vaikka verkkosivustojen personointia on tutkittu verrattain pitkään, käytännön
hyödyntäminen toimialalla on vielä harvinaista~\cite{viite?}.
Verkkosivuston ulkoasu suunnitellaan nykyäänkin lähtökohtaisesti ihmisen toimesta,
ja myös ulkoasusuunnitelman tulkitseminen ja ohjelmoiminen lopulliseksi
verkkosivustoksi on manuaalista työtä. Käyttäjille tai edes käyttäjäryhmille ei
siis ole kustannustehokasta suunnitella personoituja versioita sivustoista.
Yhden ja saman version jakelu kaikille verkkosivuston käyttäjille ei ole
kuitenkaan aina optimaalista, sillä käyttäjillä on usein eri tarpeita muun
muassa kulttuuritaustasta ja iästä riippuen~\cite{viite?}.

Työn tarkoitus on selvittää mitä menetelmiä verkkosivuston ulkoasun eri osien
personointiin on olemassa. Tarkasteltavia osia ovat muun muassa verkkosivuston
asettelu, kirjasin- ja värivalinnat sekä valikot. Työ tarkastelee myös eri
personointimenetelmien toimintatapoja. Monet tarkasteltavista menetelmistä
perustuvat matemaattisen optimointiin, mutta hyödyntävät myös
käyttöliittymäsuunnittelun heuristiikkoja kuten tekstin luettavuuskriiterejä
tuloksissaan. Oma lukunsa ovat koneoppimiseen perustuvat menetelmät, joiden
opetusdatana on käytetty esimerkiksi olemassa olevia verkkosivustoja. Työ myös
vertailee menetelmien käyttöönoton kustannuksia saavutettaviin hyötyihin, ja
pyrkii tätä kautta löytämään matalan kynnyksen persononointimenetelmät, joista
saa suurimman hyödyn.

\clearpage

\section{Aikaisempi tutkimus}

Tarvitaanko tätä? Tulee mun mielestä ilmi seuraavissa kappaleissa.

\clearpage

\section{Web- eli verkkosivujen kehitys}

Verkkosivujen kehitys on muuhun ohjelmistokehitykseen verrattuna nuori ala ja
edelleen jatkuvassa murroksessa. Tässä kappaleessa esitellään web-kehityksen
nykytilannetta työn taustatiedoksi.

\subsection{Web-suunnittelu}

Web-suunnittelu eli verkkosivujen suunnittelu on käytännön tasolla lähellä muita
graafisen suunnittelun aloja kuten printtisuunnittelua, vaikka mediana web
tarjoaa paljon enemmän mahdollisuuksia vuorovaikutukseen. Web-suunnittelun
päätuotos eli arkikielessä \textit{leiska} on ei-vuorovaikutteinen vektorikuva
siitä, miltä verkkosivuston tulee näyttää. Leiskan lisäksi suunnittelija tuottaa
usein rajoitetun interaktiivisia prototyyppejä, joilla sivuston toimintaa voi
esitellä leiskan kautta esimerkiksi asiakkaalle ennen toteutusta. Leiskaan
tehdään tarvittavat mukautukset eri asiakasryhmille vain ennalta sovittujen
määrittelyjen ja palvelumuotoilun pohjalta. Yleisin tällainen mukautus on omien
leiskojen tuottaminen eri laitetyypeille, kuten mobiili-, tabletti- ja
työpöytälaitteille. Laitetyyppeihin perustuvaa mukauttamista kutsutaan
responsiiviseksi web-suunnitteluksi, josta kerrotaan lisää
luvussa~\ref{responsive-web-design}. Varsinaista personointia leiskoihin ei
yleensä sisällytetä.

\subsection{Web-kehitys}

Web-kehittäjän tehtävä on kääntää leiska toimivaksi verkkosivustoksi. Leiska ei
ei-vuorovaikutteisesta luonteestaan huolimatta ole nykyään enää pelkkä
rasterikuva, toisin kuin yleisesti vielä 2000-luvun alussa. Vektorimuotoiseen
leiskaan on upotettu kaikki web-kehittäjän tarvitsemat yksityiskohdat ulkoasun
toteuttamiseen, kuten kirjasintyyppi ja -koko, marginaalit ja värikoodit.
Web-kehittäjä hyödyntää työssään joko samoja työkaluja kuin suunnittelija, tai
varta vasten leiskojen käsittelyyn tarkoitettuja kehittäjätyökaluja.

Karkeasti yleistäen web-kehityksessä on pääosassa kolme teknologiaa: HTML, CSS
ja JavaScript. HTML-merkintäkielen avulla määritetään verkkosivun rakenne ja eri
osien rakenteellinen merkitys. Selain ymmärtää eri osien rakenteellisen
merkityksen ja pystyy piirtämään ne sivulle oikealla tavalla. CSS-säännöstön
avulla määritetään verkkosivun osien tyyli ja asettelu, eli se on
HTML-merkintäkielen lisäksi merkittävässä osassa leiskaa toteuttaessa.
JavaScript on ainoa web-alustalla kattavasti tuettu varsinainen
ohjelmointikieli. Sillä on mahdollista luoda vuorovaikutusta käyttäjän ja
sivuston välille.

Nykyään web-kehittäjä rakentaa sivustoja harvemmin enää pelkästään näillä
kolmella perusteknologialla, vaan niiden päälle on kehitetty teknologioita ja
kirjastoja jotka vähentävät toistuvaa työtä ja huolehtivat jossain määrin
hyvistä käytänteistä kuten saavutettavuudesta. Nämä teknologiat hyödyntävät
enenevissä määrin JavaScript-ohjelmointikieltä, mikä on tuonut web-kehitystä
lähemmäs perinteisempää ohjelmistokehitystä. Aiemmin verkkosivustoja oli yleistä
rakentaa lähes ilman ohjelmointityötä pelkän HTML-merkintäkielen ja
CSS-tyylisäännöstön avulla. Sivustojen käytettävyyttä parannettiin kevyen
JavaScript-skriptauksen kera siellä ja täällä, mutta varsinainen ohjelmointityö
rajoittui yleensä palvelinpuolelle. Palvelinpuolen kehitystä ei tässä työssä
esitellä tai käsitellä työn rajauksesta johtuen.

Sittemmin tärkeä edistysaskel web-kehityksessä on tapahtunut 2010-luvun
loppupuolen jälkeen, kun JavaScript-käyttöliittymäkirjastot ovat yleistyneet
räjähdysmäisesti. Yleisimmät käyttöliittymäkirjastot vuonna 2022 ovat React ja
Vue. Käyttöliittymäkirjastot abstraktoivat perinteisen web-kehityksen kuten
HTML-rakenteen luomisen, CSS-tyylittelyn ja JavaScript-toiminnallisuuden
rakentamisen semanttisten web-komponenttien taakse. Komponentit enkapsuloivat
jonkin tietyn toiminnallisuuden, ja niitä on suhteellisen helppo
uudelleenkäyttää myöhemmin. Komponentit myös mahdollistavat osaltaan
suunnittelujärjestelmien (engl.\ \textit{design system}) luomisen jo projektin
alussa. Suunnittelujärjestelmä sisältää yleensä projektin komponenttikirjaston,
joka käsittää sivuston rakennuspalikat kuten painikkeet, otsikot ja lomakkeet.

Tähän vielä selkeytystä komponenteista ja mitä muuta fronttikirjastot tekee.
Selkeytystä myös design system osalta. Jotain muuta ehkä myös, esim. valmiit
komponenttikirjastot tai frameworkit (Bootstrap, Tailwind, jne)?

\clearpage

\section{Personoinnin hyödyt ja haitat}

Personoinnilla tarkoitetaan käyttäjästä kerätyn tiedon perusteella tehtävää
automaattista mukauttamista. Kerätty tieto voi olla hyvin erilaisista lähteistä,
kuten asiointihistoriasta, julkishallinnon rekistereistä tai verkkosivujen
tapauksessa käyttäytymisestä sivustolla. Tieto jonka pohjalta personointia
tehdään voi olla myös käyttäjän itse ilmoittamaa, joskin tällöin puhutaan
yleensä kustomoinnista. Personointia voidaan tehdä missä tahansa rajapinnassa
jossa on käyttäjän ja palvelun välistä vuorovaikutusta, kuten toimipiste-,
puhelin- tai verkkoasioinnissa. Tämä työ keskittyy verkkoasioinnin personointiin
ja nimenomaan verkkosivustojen ulkoasun personointiin.

Personointi osana ihmisen ja tietokoneen välistä vuorovaikutusta on laaja aihe,
ja työ on tarkoituksella rajattu vain verkkosivuston ulkoasun personointiin
näkökulman tarkentamiseksi. Työ keskittyy pääasiassa ulkoasun personointiin
vaikka myös verkkosivuston sisällön personointia sivutaan, sillä sisällön
personointi on aiheena laaja ja kasvattaisi työn rajausta liian suureksi.
Sisällön personointiin kuuluu esimerkiksi sosiaalisen median alustojen
suosittelualgoritmit ja verkkomainonnan kohdennus, jotka eroavat menetelmiltään
merkittävästi verkkosivuston ulkoasun personoinnista. Ulkoasun personointiin
kuuluu kuitenkin olennaisena osana käyttäjälle tärkeän sisällön korostaminen,
jota tarkastellaan luvussa~\ref{layout-personalization}.

Tarve personoinnille vaihtelee verkkosivuston mukaan, mutta yleisesti voidaan
todeta että personoinnin tavoitteena on ohjata käyttäjän toimintaa
verkkosivuston palveluntarjoajan tavoitteiden mukaisesti, ja täten joko lisätä
palveluntarjoajan tuloja tai vähentää sen menoja. Esimerkiksi tyypillisen
verkossa toimivan uutismedian tavoite on kerätä tuloja näyttämällä mainoksia
artikkelien yhteydessä. Personoimalla käyttäjälle näytettäviä sisältöjä
esimerkiksi korostamalla tälle mielenkiintoisia aiheita, käyttäjä saadaan
lukemaan enemmän artikkeleja ja täten kasvatettua uutismedian mainostuloja. Sitä
vastoin esimerkiksi kunta voisi pyrkiä vähentämään asiointiin liittyviä soittoja
puhelinneuvontaan, jonka ylläpitäminen on kallista. Verkkosivustonsa etusivua
personoimalla kunta voisi nostaa juuri käyttäjälle ajankohtaiset lomakkeet heti
sivun alkuun, ja täten helpottaa käyttäjän asiointia.

Personoinnilla voidaan siis säästää rahaa, mutta sen kautta on mahdollista saada
myös yleishyödyllisempiä hyötyjä. Verkkosivuston personointi mahdollistaa
yksilöllisen tiedonvälityksen kokoluokassa, joka ei aiemmin ole ollut
mahdollista. Printtimedia skaalautuu kyllä hyvin ja sen kautta tietoa on ollut
mahdollista levittää laajalle jo vuosisatoja, mutta sen personointi on liki
mahdotonta tai ainakin erittäin kallista. Toinen ääripää eli tiedonvälitys
henkilöltä toiselle on aina personoitua, mutta se ei skaalaudu. Julkinen valta
ja muut yleishyödylliset toimijat voivat siis hyödyntää personointia
tiedonvälityksessä tavoittaakseen suurempia yleisöjä. Tästä on ollut hyötyä
esimerkiksi COVID-19 -pandemian aikana, kun rokotekampanjoita on voitu kohdentaa
verkossa tärkeille yleisöille~\cite{viite?}.

Personoinnissa on kuitenkin myös selkeitä kipukohtia. Personointi on
automaattista ja perustuu käyttäjästä kerättyyn tietoon. Tietoa kerätään
mahdollisesti monesta eri kanavasta, ja on tärkeää että tietojen keräämiseen on
aina käyttäjän suostumus. Tietolähteiden lisääntyessä käyttäjä ei välttämättä
enää ole kartalla siitä, mitä tietoa on suostunut luovuttamaan ja mihin.
Tietojen luovutuspyyntöihin myös turtuu nopeasti, mistä esimerkkinä on EUn
GDPR-direktiivin myötä yleistyneet evästesuostumukset. Suostumusta kysytään
lähes jokaisella verkkosivustolla, koska verkkoanalytiikka ja kohdennettu
mainonta on niin yleistä. Asenteet tämäntyyppisen personoinnin osalta ovat
muuttuneet negatiiviseen suuntaan, kun vielä 90-luvulla sisällön personointi
nähtiin verkkoon siirtyvän printtimedian pelastajana~\cite{adams_1995}.

Arkielämän teknistyessä käyttäjälle ei myöskään ole välttämättä aina selvää,
mitä kaikkea tietoa on edes mahdollista kerätä. Tietoja kuten hiiren liike,
näppäimistösyöte, laitetiedot ja jossain määrin myös selailuhistoria ovat
verkkosivuston käytettävissä. Laitetietojen pohjalta sivusto voi muodostaa
käyttäjästä sormenjäljen, jota on käytännössä mahdoton muuttaa tai pyyhkiä pois
vaihtamatta laitetta. Evästeiden avulla isot mainosverkostot pystyvät seuraamaan
käyttäjää sivustolta toiselle ja personoimaan mainontaa selailukäyttäytymisen
perusteella.

Personointi ei aina ole edes haluttua. Monet alustapohjaiset teknologiayritykset
ovat viime vuosina panostaneet erilaisiin suosittelualgoritmeihin kohdentaakseen
sisältöään paremmin. Suosittelualgoritmit toimivat hyvin suurelle osalle
käyttäjäkunnasta, mutta saattavat hankaloittaa palvelun käyttöä pienelle osalle,
johon kuuluvat eivät istu hyvin algoritmin luokittelijaan. Suosittelualgoritmit
ovat myös korvanneet palveluiden aiemmin käyttäjälle tarjoamia
kustomointimahdollisuuksia kuten sisällön järjestelyasetuksia, mikä on
hankaloittanut palvelun käyttöä~\cite{patel_2022}.

Jo aiemmin mainittu COVID-19 rokotetietoisuuden kohdentaminen on myös
kaksiteräinen miekka. Jos hyväntahtoiset toimijat kuten terveysvirastot pystyvät
personoimaan tiedonvälitystä avainryhmille, sama onnistuu myös pahantahtoisilta
toimijoilta. Välittämällä kyseenalaista tietoa kuten salaliittoteorioita
vastaanottavaisille ryhmille voidaan lyödä kiilaa yhteiskunnan sisälle.

Tämän työn aihe eli verkkosivujen ulkoasun personointi on kuitenkin verrattain
harmiton personoinnin osa-alue. Sen haittapuolena on lähinnä lisääntynyt
tekninen monimutkaisuus käytetystä menetelmästä riippuen. Saavutetut hyödyt ovat
verkkosivuston palveluntarjoajan lisääntyvässä asiakastyytyväisyydessä ja sitä
kautta lisääntyvissä tuloissa tai vähentyvissä menoissa.

\clearpage

\section{Asettelun personointi}\label{layout-personalization}

Sivun asettelu määrittää verkkosivun perusrakenteen ja sivun sisällön
keskinäisen järjestyksen. Tyypillisesti verkkosivuston yksittäisellä sivulla on
vähintään ylätunniste (engl.\ \textit{header}), sisältöalue (engl.\
\textit{content area}) ja alatunniste (engl.\ \textit{footer}). Ylätunniste ja
alatunniste pysyvät yleensä suurin piirtein muuttumattomina sivustolla
navigoidessa, mutta sisältöalueen sisältö luonnollisesti vaihtuu sivun mukaan.
Ylätunniste ja alatunniste ovat valikonomaisia sivuston osia, joita käsitellään
luvussa~\ref{menus}. Ylätunniste sijaitsee aina sivun alussa ja alatunniste
sivun lopussa. Niiden asettelun personointi ei ole mielekästä, joten niitä ei
tarkastella enempää tässä luvussa.

Sisältöalueen asettelu perustuu nykyään lähtökohtaisesti ruudukkorakenteeseen
(engl.\ \textit{grid layout}). Ruudukko koostuu sivustosta ja päätelaitteen
näyttökoosta riippuen yleensä n. \ 1--16 sarakkeesta (engl.\ \textit{column}),
jotka on erotettu toisistaan vakiosuuruisella marginaalilla (engl.\
\textit{margin}). Sarakkeiden leveys, joka on kääntäen verrannollinen
sarakkeiden määrään, perustuu suhteelliseen osaan saatavilla olevasta
ruudukkoalueen (engl.\ \textit{grid container}) leveydestä, ja sarakkeiden
leveys muuttuu yhdessä ruudukkoalueen kanssa. Ruudukkoalueelle sijoittuvat
elementit (engl.\ \textit{element}), kuten tekstipalstat ja kuvat, voivat kattaa
yhden tai useamman sarakkeen leveyden. Elementtien leveys määritetään kuitenkin
nimenomaan sarakkeiden kautta, eivätkä ne voi vapaasti laajentua leveyssuunnassa
niiden sisällön kasvaessa. Sen sijaan elementit laajenevat tarvittaessa
pituussuunnassa. Tyypillisesti verkkosivut eivät mahdukaan kokonaan kerralla
näytölle, vaan niitä pitää vierittää alaspäin jolloin lisää sisältöä tulee
näkyville.

\subsection{Responsiivinen web-suunnittelu}\label{responsive-web-design}

Ennen 2010-lukua asettelu tehtiin lähtökohtaisesti staattisesti, tarkoittaen
että sivun sisältöalueella oli ennalta määrätty leveys. Leveys oli yleensä
maksimissaan 1024 pikseliä, joka oli senaikaiset päätelaitteet huomioon ottaen sopiva
koko~\cite{viite?}. Sisältöalueen leveys ei muuttunut päätelaitteen mukaan. Sivun suunnittelu
ja toteutus oli verrattain helppoa, koska hankalista asettelun rajatapauksista
ei tarvinnut huolehtia. Toisaalta staattisesti rakennetut sivustot alkoivat olla
hankalia käyttää älypuhelinten yleistyessä, sillä sivustojen käyttö pienellä
näyttökoolla vaati zoomausta sisällön erottamiseksi.

Nykypäivänä on tyypillistä, että sivuston asettelu mukautuu päätelaitteen
näyttökokoon. Näyttökokoon perustuvaa asettelun mukautumista kutsutaan
responsiiviseksi web-suunnitteluksi (engl.\ \textit{responsive web design}),
joskin termi kattaa myös muita mukauttamisperusteita, kuten paperille
tulostuksen. Sivuston perustuessa ruudukkorakenteeseen asettelu mukautuu
automaattisesti näyttökoon muutoksiin, sillä sarakkeiden leveys on määritetty
suhteessa ruudukkoalueeseen. Asettelua voi kuitenkin parantaa entisestään
lisäämällä sarakkeiden määrää näyttökoon kasvaessa, ja vähentämällä sarakkeiden
määrää näyttökoon pienentyessä. Tällä tavalla esimerkiksi työpöytälaiteella
vierekkäin näkyvät elementit voidaan mobiililaitteella asetella allekkain,
jolloin ne ovat pienelläkin näyttökoolla yhtä suuria kuin työpöytälaiteella.

Asettelun muuttaminen näyttökoon muuttuessa tapahtuu leveyteen perustuvien
pysäytyspisteiden (engl.\ \textit{breakpoint}) avulla. Pysäytyspisteet
määritetään \textit{media queries} -tekniikan~\cite{Rivoal:12:MQ} avulla, joka
mahdollistaa muun muassa CSS-tyylien kohdentamisen päätelaitteen ominaisuuksien
mukaan. Pysäytyspisteisiin pohjautuva asettelun mukauttaminen on yleinen ja
helppo tapa mukauttaa sivuston asettelua eri päätelaitteille. Pysäytyspisteiden
avulla asettelu mukautetaan yleensä vähintään niin, että puhelimille,
tableteille ja työpöytälaitteille on kullekin oma asettelunsa.

\subsection{Suunnittelijan työkalut}

Personointimenetelmät voivat toimia suunnittelijan työkalujen kontekstissa.
Tällöin ne on helpompi ottaa käyttöön, koska työkalut on lähtökohtaisesti samat
kaikilla ja niihin integroituminen voidaan lisäosa-tyyppisten ratkaisujen
kautta. Nykyään suosittuja verkkosivustojen suunnitteluun käytettäviä
ohjelmistoja ovat muun muassa Sketch, Figma ja Adobe XD, joihin kaikkiin on
mahdollista rakentaa lisäosia. Perinteisempiä työkaluja edustavat Adoben
tuoteperheen Photoshop ja Illustrator, joiden käyttö web-suunnittelussa on
vähentynyt huomattavasti.

Yksi suunnittelijoille suunnattu apukeino on Pang et al.~\cite{Pang2016}
esittelemä menetelmä verkkosivuston asettelun optimoimiseen käyttäjien
silmänliikkeistä kerätyn datan perusteella. Suunnittelija kertoo työkalulle
tavoittelemansa elementtien lukujärjestyksen, ja työkalu tuottaa suunnitelman
verkkosivusta jossa elementtien asettelu on optimoitu annetun lukujärjestyksen
mukaan. Pangin työkalu ei integroidu suoraan suunnittelijan työkaluihin, mutta
vastaavan kaupallisen ratkaisun tuominen yleisimpiin työkaluihin olisi varmasti
tervetullut apu suunnittelijoille.

Suunnitteluvaiheen optimointimenetelmien kautta on hankala personoida
verkkosivuja, koska toteutusvaihe vaatii edelleen manuaalista työtä. Tietyllä
tapaa Pangin työkalun kaltaiset menetelmät voidaan nähdä personoivan leiskaa
itse suunnittelijalle, mutta tämä menee hiukan saivartelun puolelle.
Personointimenetelmät jäävätkin suunnittelijan työkalujen osalta responsiivisen
web-suunnittelun tapaan melko yleiselle, isoille käyttäjäryhmille suunnatulle
tasolle.

\subsection{Personointi optimointiongelmana}

Responsiivisen web-suunnittelun haasteena on se, että se täytyy ottaa huomioon
jo sivuston suunnittelu- ja toteutusvaiheessa. Suunnittelijan tulee prosessin
aikaisessa vaiheessa päättää, mitkä pysäytyspisteet sivustolla on käytössä.
Pysäytyspisteiden muuttaminen myöhemmin suunnittelun tai toteutuksen aikana on
hyvin työlästä, sillä pysäytyspisteet määrittävät asettelun perustan ja niiden
muuttaminen vaikuttaa siten kaikkeen muuhun.

Responsiivinen web-suunnittelu ei ole personointia sanan varsinaisessa
merkityksessä, sillä sivuston responsiviisuus on määritelty jo
suunnitteluvaiheessa eikä aidosti mukaudu yksittäisen käyttäjän tarpeisiin.
Responsiivisen web-suunnittelun kautta tapahtuva asettelun personointi perustuu
vain käyttäjän laitteen karkeaan kategorisointiin, yleensä joko mobiili-,
tabletti- tai työpöytälaitteeksi.

Näistä lähtökohdista tutkimusta on kohdistunut laskennallisten menetelmien
hyödyntämiseen responsiivisessa web-suunnittelussa. Verkkosivusto on
pohjimmiltaan rakenteellista sisältöä, joten sitä voidaan käsitellä
koneellisesti. Jos responsiivinen web-suunnittelu kyetään automatisoimaan,
vähentää se suunnittelijan ja kehittäjän työtaakkaa huomattavasti. Tällöin
riittää että sivustosta suunnitellaan vain mobiiliversio (ns. \textit{mobile
first} web-suunnittelu), josta voidaan laskennallisten menetelmien avulla
laajentaa muille laitteille sopivat versiot.

Aalto-yliopiston \href{https://userinterfaces.aalto.fi/}{User Interfaces
-tutkimusryhmä} kehitti vuonna 2020 \textit{Layout as a
Service}-alustakonseptin~\cite{laine2020_laas} ja myöhemmin vuonna 2021 siihen
pohjautuvan \textit{Computational responsive web design
(C-RWD)}-teknologiakonseptin~\cite{laine2021responsive}. Tutkimusryhmän C-RWD
-teknologia kykenee automatisoimaan sivuston mukauttamisen eri päätelaitteille
ja näyttöko'oille. Teknologia seuraa sivuston vieralijoiden käyttäytymiseen
liittyviä tietoja, kuten vierailuaikaa kullakin sivulla ja mitä sisältöjä on
klikattu. C-RWD parsii sivun HTML-rakennetta ja kykenee kerättyjen tietojen
avulla optimoimaan siitä käyttäjälle personoidun version. Personoitu versio
sivustosta tarjoillaan käyttäjälle vanhan sivuston sijasta. Käyttäjälle
tärkeiksi arvioituja sisältöjä voidaan esimerkiksi nostaa ylemmäs sivulla,
jolloin ne näkyvät käyttäjälle heti sivun latautuessa.

C-RWD laajentaa asettelun laskennallista suunnittelua nimenomaan responsiivisen
web-suunnittelun osalta. Aiempaa tutkimusta edustaa mm.
SUPPLE~\cite{10.1145/964442.964461}, joka tutki sivun asettelua
optimointiongelmana. SUPPLE kuitenkin vaati suunnittelijaa määrittämään
asettelun rajoitteet, kuten päätelaitteen näyttökoon ja käyttötavan. C-RWD
päättelee optimointiin tarvittavat rajoitteet itse seuraamalla käyttäjän
käyttäytymistä.

\section{Ilmeen personointi}

Ilmeellä tarkoitetaan tässä verkkosivuston visuaalista tyyliä. Siihen sisältyy
käytetty kirjasintyyppi (engl.\ \textit{font}) ja muut tekstiin liittyvät
ominaisuudet kuten kirjasinkoko sekä lihavoinnin tai kursiivin kaltaiset
muotoseikat. Ilmeeseen sisältyy myös verkkosivuston väripaletti, joka on hyvin
tärkeässä osassa personoinnissa. Väreisiin liittyy tunnelatauksia ja täten
värivalinnat vaikuttavat siihen, millaisena ihmiset kokevat verkkosivuston
käytön. Ilmeen kannalta oleellista on myös tyhjän tilan käyttö eli se kuinka
paljon tilaa sivuston osien välillä on. Nämä ovat kaikki asioita, jotka voi
jossain määrin yleistää kaikkeen graafiseen suunnitteluun web-suunnittelun
lisäksi, ja ilmeen personointia on tutkittu myös graafisen suunnittelun
näkökulmasta.

\subsection{Tekstin ilme}

Luettavuus, eri kielissä eri aakkoset, miten pelaa yhteen tekstin sävyn kanssa
(virallinen teksti ei voi olla comic sans jne)

Voiko esim https://www.metaflop.com/ käyttää automaattiseen personointiin?

Tähän O'Donovanilla hyvää asiaa

\subsection{Värit}

Väripaletin suunnittelu on aina hankalaa. Verkkosivustolla on usein käytössä
ainakin yksi pääväri, yksi huomioväri ja lisäksi muutama hillitympi sävy.
Suunnittelija valitsee värit intuition ja heuristiikkojen kuten värin `lämmön'
perusteella~\cite{odonovan_2015}. Onneksi suunnittelijan ei yleensä tarvitse
aloittaa tyhjästä, vaan ennen verkkosivuston suunnittelua on jo olemassa
jonkinlainen organisaation kattava suunnittelujärjestelmä.

Haastavaa värisuunnittelua on mahdollista helpottaa personoimalla väripalettia
kullekin käyttäjälle sopivaksi. Personoinissa täytyy toki ottaa huomioon
brändiin liittyvät seikat, mutta esimerkiksi huomiovärin kaltaisia
yksityiskohtia on mahdollista personoida hyvin tuloksin. Suhtautuminen väreihin
on erilaista eri puolilla maailmaa. Jos verkkosivuston käyttäjäkunta koostuu
vain pienen kultuurillisesti yhtenäisen alueen asukkaista, värien personoinnista
ei välttämättä ole hyötyä. Jos käyttäjäkunta on laajempaa tai jopa globaalia,
väripersonointi nousee uuteen merkitykseen. Reinecke et
al.~\cite{10.1145/2556288.2557052} tutkivat värien merkitystä web-suunnittelussa
pyytämällä ihmisiä eri kulttuuritaustoista arvioimaan eri verkkosivustojen
visuaalista miellyttävyyttä. He huomasivat, että esimerkiksi makedonialaiset
pitävät keskimäärin värikkäämistä sivuista ja toisaalta venäläiset suosivat
yksinkertaisempia leiskoja ja väripaletteja.

Jatkuu... mitä menetelmiä värien personointiin? pitääkö ottaa huomioon jo
suunnitteluvaiheessa vai voidaanko automatisoida?

\subsection{Kuvat ja grafiikat}

Tähän ainakin Personalization of image enhancement artikkelista asiaa.

\subsection{Tyhjän tilan käyttö}

Mainitaanko golden ratio? Mitä muuta?

\section{Valikoiden personointi}\label{menus}

Valikkojen uudelleenjärjestely jne.

\section{Muu personointi}

Lyhyesti vaan. Kannattaakohan ees pitää?

\subsection{Sisältö}

Content-based filtering, etc.

\subsection{Lomakkeet}

Lomakkeiden personointi.

\subsection{Saavutettavuus}

Apuelementit (tooltips), aria, jne.

\clearpage

\section{Personointimenetelmien vertailu}

Kuinka personoinnin menetelmät voidaan luokitella niin, että ne ovat helposti
vertailtavissa verkkosivuston kehittäjän näkökulmasta?

Tähän olisi hyvä saada grafiikkaa

\subsection{Käyttöönoton monimutkaisuus ja hinta}

Kuinka kallista / vaikeaa ne on ottaa käyttöön? Onko avoimia haasteita? Onko
tilanteita joissa menetelmiä ei voi käyttää?

\subsection{Vaikuttavuus ja hyöty}

Kuinka suuri positiivinen vaikutus niillä on?

\subsection{Nykyinen käyttö toimialalla}

Kuinka laajasti ne ovat käytössä toimialalla?

\subsection{Tulevaisuuden näkymät}

Onko jatkokehitystä, "trendaako"?

\clearpage

\section{Yhteenveto}

Summa summarum parhaat menetelmät, jotka voi ottaa käyttöön matalalla
kynnyksellä.

\clearpage

\thesisbibliography{}
\printbibliography{}

\clearpage
\thesisappendix{}

\section{Esimerkki liitteestä\label{LiiteA}}

- Kuva tyypillisestä verkkosivusta jossa korostettu työn kannalta relevantit
osat, kuten rakenne, ilme, sisältö, saavutettavuus.

\end{document}
